% Created 2020-04-14 Tue 13:07
% Intended LaTeX compiler: pdflatex
\documentclass[11pt]{article}
\usepackage[utf8]{inputenc}
\usepackage[T1]{fontenc}
\usepackage{graphicx}
\usepackage{grffile}
\usepackage{longtable}
\usepackage{wrapfig}
\usepackage{rotating}
\usepackage[normalem]{ulem}
\usepackage{amsmath}
\usepackage{textcomp}
\usepackage{amssymb}
\usepackage{capt-of}
\usepackage{hyperref}
\usepackage{ctex}
\author{hyliu}
\date{\today}
\title{}
\hypersetup{
 pdfauthor={hyliu},
 pdftitle={},
 pdfkeywords={},
 pdfsubject={},
 pdfcreator={Emacs 26.3 (Org mode 9.2.6)}, 
 pdflang={English}}
\begin{document}

\tableofcontents

\section{核方法}
\label{sec:orgd514c1f}
\subsection{核函数的定义}
\label{sec:org320cc98}
假设 \(\mathcal{X}\) 为输入空间, \(\mathcal{H}\) 为特征空间,如果存在一个从 \(\mathcal{X}\) 到 \(\mathcal{H}\) 的映射
$$\phi(x): \mathcal{X} \rightarrow \mathcal{H}$$
使得对所有 \(x, z \in \mathcal{X}\) , 函数 \(K(x, z)\) 满足以下条件
$$K(x, z) = \phi(x) \cdot \phi(z)$$
则称 \(k(x, z)\) 为核函数,\(\phi(x)\) 为映射函数,\(\phi(x) \cdot \phi(z)\) 为 \(\phi(x)\) 与 \(\phi(z)\) 的内积。
\subsection{核函数的作用}
\label{sec:orgbfad47a}
:x: 
\subsection{}
\label{sec:org275c200}
\end{document}